\droppedcapital{T}{raining} supervised models for personal attribute prediction requires significant amounts of labeled data. Although there is plenty of available dialogues for training chat-bots, the speakers' profiles containing personal information are rarely accessible. There is only a limited number of publicly available conversational datasets labeled with speakers' attributes, mostly containing only mere basic demographic facts, like \textit{gender} or \textit{age}. To efficiently train models for inferring rich user profiles we create large scale datasets containing a wide range of personal attributes.

Our work is focused on personal attribute prediction from the \textit{textual} representation of the dialogues. We can distinguish different sources of conversational data in textual format: spoken dialogue transcripts, private messages and emails, posts in web discussion forums, etc. In this chapter we describe our work on collecting and labelling \textit{(i)} transcribed dialogues, and \textit{(ii)} social media submissions. All discussed datasets are available at \href{http://pkb.mpi-inf.mpg.de}{http://pkb.mpi-inf.mpg.de}.

%Although people often reveal a lot of personal facts in what they say, still inferring these facts from the textual format of data is hard. For transcribed dialogues, a lot of cues are missed, by discarding the visual information and intonation variations. Similarly, social media submissions contain additional meta information (like friendship network or number of likes), which can aid prediction. Yet, we work with a more general case, assuming that additional data might not be available (for instance, when an intelligent assistant creates a PKB from a transcribed history of conversations).

