
\section{Background}
\label{background}

Interactions between people can have multiple fine-grained features, describing various aspects of their communication. For example, interactions can be characterized by attachment style (such as \textit{commitment} or \textit{avoidance}) \cite{qamar2021relationship} or power hierarchy (\textit{subordinate} or \textit{superior}) \cite{prabhakaran2014predicting}. There are ample related social studies researching these characteristics; yet, there is no formal ontology for them \cite{rashid2018characterizing}. One way to organize the features of interpersonal interactions was proposed by \cite{rashid2017dimensions}, defined as \textit{dimensions} of the relationships.

Most of the relationships that we defined for our experiments also have particular interpersonal characteristics. For example, the \textit{enemy} relationship can be described as \textit{competitive} as opposed to \textit{cooperative}; the relationship between \textit{parent} and \textit{child} is in most cases \textit{intimate}. Thus, we find it beneficial to enhance the relationship prediction model with the known features of the speakers' interactions. 

We use the definition of \textit{interpersonal dimensions} \cite{wish1976perceived} of speakers' interactions and relationships, following classification of \citet{rashid2018characterizing}, which we used as an additional input to our model. We note that the discussed interpersonal dimensions are descriptive of the relationship between a particular pair of speakers, but not of the relationship type in general (for example, the interaction between \textit{colleagues} can be both \textit{cooperative} and \textit{competitive}). However, in general any interpersonal dimension can be more or less typical for a relationship type; we use this information to give the model hints about applicable predictions.

\citet{rashid2018characterizing} consider 11 interpersonal dimensions, divided into dimensions of relationships and interactions, as shown in Table \ref{dimensions}. In our model we use all proposed dimensions to provide a comprehensive summary of the relationship's fine-grain characteristics. %For details on each dimension we refer the reader to the original paper \cite{rashid2018characterizing}.
\citet{rashid2018characterizing} also provide a conversational dataset, where every utterance has annotations for each considered interpersonal dimension. We utilize this dataset to pretrain a model for utterance-level dimension classification and create separate representations for each dimension, which are later used in PRIDE.

\begin{table}[t!]
\centering
\begin{adjustbox}{width=0.7\textwidth}
\begin{tabular}{@{}l|ll@{}}
\multirow{4}{*}{relationships} & cooperative vs. noncooperative & equal vs. hierarchical\\ 
 & pleasure vs. work oriented & intense vs. superficial \\ 
 & intimate vs.unintimate & active vs. passive \\ 
 & temporary vs. long term \\ \hline
\multirow{2}{*}{interactions}  & cooperative vs. noncooperative & active vs. passive \\ 
 & concurrent vs. non concurrent & near vs. distant \\                                                          
\end{tabular}
\end{adjustbox}
\caption{Interpersonal dimensions used in PRIDE.}
\label{dimensions}
\end{table}


