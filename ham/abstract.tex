%personal information in dialogues is often implicit and must be inferred.
%There have been only limited attempts to perform IE on personal conversations, however, motivating us to
%
% Information extraction (IE) is the methodology for distilling structured data,
% typically in the form of subject-predicate-object triples, out of natural-language text.
% It has been successfully applied to a variety of text genres such as social media posts, 
% scientific publications and Wikipedia articles, to power applications like
% sentiment analysis and knowledge base population.
% However, there have been only few attempts and no compelling ones
% to perform IE on personal conversations.
%
\droppedcapital{O}pen-domain dialogue agents must be able to converse about many topics while incorporating knowledge about the user into the conversation.
The background information about user's demographics, such as age or gender, can help the chat-bots adjust their conversational style, make relevant recommendations and initiate engaging discussions. Instead of asking the users to manually provide their personal information or seek it in the external sources, we propose to directly extract such facts from the user's dialogues. This problem is more challenging than the established task of 
information extraction from scientific publications or Wikipedia articles, because dialogues
often give merely implicit cues about the speaker.

We propose methods for inferring personal attributes, such as
profession, age or family status, from conversations using deep learning.
Specifically, we propose several
\textit{Hidden Attribute Models}, which are 
neural networks 
leveraging 
attention mechanisms and embeddings. 
Our methods are trained on a per-predicate basis to output
rankings of object values for a given subject-predicate combination
(e.g., ranking the doctor and nurse professions high when speakers talk
about patients, emergency rooms, etc).
Experiments with various conversational texts including Reddit discussions, movie scripts and a collection of crowdsourced personal dialogues
demonstrate the viability of our methods and their superior performance compared
to state-of-the-art baselines.
