\documentclass[11pt, oneside]{book}
\usepackage[utf8]{inputenc}


\begin{document}

\chapter{Conclusion}

This thesis is concerned with predicting personal knowledge from conversations. Such information can be used to populate a personal knowledge base, enhancing many downstream applications. The ambiguity of conversational utterances makes it challenging to automatically process them; thus the task of speakers' attribute inference is underexplored in related work. In our research we overcome the limitations of the prior studies, proposing the models which can accurately predict a wide range of personal facts.

%We explore inferring personal facts from dialogue data, which is abundant but challenging to automatically process.  Given the challenging nature of the conversational data, inferring speakers' attributes is particularly difficult and is currently underexplored in the related work. Yet, the knowledge of such attributes can augment many downstream applications. In our research we make a step towards creating models capable of predicting personal information.

In Chapter 4 we described \textit{Hidden Attribute Models} (HAMs), capable of predicting speakers' demographic attributes: \textit{age}, \textit{gender}, \textit{profession} and \textit{family status}. HAMs utilize hierarchical conversational structure, making precise predictions at low computational costs. We have shown the capacity of HAMs to transfer learn among different conversational datasets, which is essential for applying the model to the real-life scenarios.

In Chapter 5 we presented \textit{Conversational Hidden Attribute Retrieval Model} (CHARM), designed for predicting the values of the long-tailed \textit{profession} and \textit{hobby} attributes in a zero-shot setup. We propose a novel model design, which incorporates external knowledge to detect personal attribute values absent from the training data. CHARM makes predictions extracting keywords from the users' utterances, which ensure model's interpretability. 

In Chapter 6 we introduced \textit{PRIDE}, a model for \textit{Predicting Relationships In Dialogue Excerpts}. Unlike most prior studies, PRIDE predicts fine-grained directed relationships, which are often ambiguous even for the human evaluators. We show that blending in additional signals, such as speakers' demographic attributes, can significantly improve interpersonal relationship inference.

Additionally, to support our experiments we issued several conversational datasets, described in Chapter 3. Our datasets cover multiple personal attributes, based on the dialogues in the movies and interactions on the social media platform, providing diverse inputs for the models. Our labeling strategies and manual verification ensure high-precision of the provided data, which will be highly valuable for further research and practical applications.

\section{Future research directions}

The research in this dissertation is only an initial step for a comprehensive and accurate prediction of personal information from conversations. In this section we list possible directions for further investigations, which we find essential for building practical and user-friendly personalized systems.

\paragraph{Open-ended attributes.} Topical and user-oriented chat-bot recommendations require the knowledge of many open-ended personal attributes, e.g. \textit{favorite singer}. It is infeasible to enumerate all possible values for such attributes, especially given that new values constantly emerge. Predicting such facts might require dedicated unsupervised extraction methods.

\paragraph{Continuous incremental predictions.} An important aspect of conversational data is that the input utterances are spread in time, arriving as conversation proceeds. The utterances might contain contradictory cues, reflecting the change in the user's preferences (or even the user's demographics). Keeping an up-to-date state of the personal knowledge base as the conversation proceeds is an important issue, which can be addressed by learning personal facts from conversations incrementally.

\paragraph{Evaluating third party information.} All our proposed models make predictions about a speaker (or speaker pair) based on their own utterances. However, a significant amount of information can be obtained by capturing the input from other conversation participants. Ideally, the model should be able to capture both the cues from subject's direct  interlocutor (``you must be coming from your shift at the hospital'') and from a third person, when the subject is not even present in the current conversation (``Roger is doing a lot of overtime in the hospital recently''). 

\paragraph{Utilizing speaker network.} Building up on the previous point, we propose that the predictions of multiple personal attributes and relationships can be made simultaneously for a group of speakers, either within a current conversation or across multiple dialogues. A good example when such approach can facilitate predictions is utilizing the dependency of interpersonal relationships (from the fact that A and B are children of C one can infer that A and B are siblings). We envision that joint inference for all conversation participants can be performed with graph methods, which enable information sharing between the speakers (graph nodes).

\paragraph{Privacy issues.} Personal attributes is a sensitive information, the exposure of which can be harmful for the end user. Our proposed models supply the evidence for their predictions, which provides the pointers to what user content was disclosing. We suggest that more research can be done into using this evidence to protect the users' personal data. 

\end{document}
